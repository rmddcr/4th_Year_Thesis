\addcontentsline{toc}{chapter}{Abstract}
\begin{center}
\textbf{\Large{Abstract}}
\end{center}
\vspace{0.7cm}


 
\vspace{12pt}


In the past, only a handful of research was done in the area of Wifi Ad-Hoc and Cellular Hybrid Networks. The main problem with these networks is their unreliability. In the current time, the cellular networks have become more stable and much cheaper so this gives us an opportunity to use the cellular network as the control plane and backup network alongside with the Wifi Ad-Hoc network in order to provide a better solution.

\vspace{12pt}

 The proposed architecture consists of two network interfaces cellular and Wifi Ad-Hoc in a multiplexing manner to provide a hybrid network connection. Every device consists of a demon(a simple program that monitors the state of the mobile devices connected to the network) that calls a server giving metadata about the node of the network. The server is responsible for managing the routing information of the active nodes in the network using the cellular control plane. The actual data transmission happens in the wifi ad-hoc network. The network returns to its initial state if one of the networks get disconnected. 
 
\vspace{12pt}
 
 A proof of concept network was implemented to explore how emerging technologies can be used to implement this mobile Ad-Hoc networking solution and to find out the implications of creating such a system in the end-users perspective.

\vspace{12pt}

 When comparing the test results gathered it was evident that there were some desirable qualities like no-cost peer to peer connections, low latency single-hop connection and above-average power efficiency. But the downside to this is that latency increase and reaction time increase when using multihop Wifi Ad-Hoc connections for devices to device communication.


\vspace{0.7cm}




\textbf{Keywords}: Hybrid Network, Ad-hoc Network, Mobile Ad-hoc network, Android 
