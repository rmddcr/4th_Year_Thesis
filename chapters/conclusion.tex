\chapter[Conclusion]{Conclusion}

\section{Introduction}
In conclusion, we can say that there are some advantages when using this hybrid Network instead of a general one. This network is particularly useful if we have less number of hops in the Ad hoc network. When comparing with the other similar networks using the relevant comparisons matrices we can say that hybrid solution has performed well in these fields.

Due to the protocol overhead, the network switching time for this network is somewhat high. This can be reduced if we do protocol implementation in the OS or Kernel level.

And the power consumption of the system is in the range of the other networks. This is a good trend as most of the hybrid networks tend to consume a higher amount of power than the other infra-structured networks. 

\vspace{0.7cm}

\section{Research questions(aims and objectives)}
\vspace{0.7cm}
By implementing a working solution of this Ad-hoc cellular hybrid solution we were able to answer the first part of the research question. Using the working prototype we showed the is feasible to implement a hybrid network using the cellular network as its control plane and Ad hoc network as its data plane. 

As you can see from the implementation chapter we found that different newly emerged technologies were used in implementing this network. Even though it is hard it is possible to create such a network using the currently available technologies.

\clearpage

So it can conclude that the hybrid solution we introduce in this research is a valid one when considering chapter 5 and have visible gains to the user from this kind of network.

\vspace{0.7cm}


\section{Limitations}

In this research, the working prototype was implemented using the Nexus 5x devices wish runs on Android OS and Linux kernel. As we adhered to this configuration we were unable to find the applicability of this methodology and techniques to other platforms of mobile devices.

And the testing was done using a limited number of Android devices which can affect the end results. So a wider range of test data is needed in order to examine the quality of the systems further. 
Due to Android kernels networking implementation being a non-standard one we were unable to change the firewall and IP rules in order to fully implement the hybrid networking protocol in the Android devices. So we demonstrate that part instead of using  Linux machines.




\vspace{0.7cm}

\section{Implications for further research}

We can extend the implementation to multiple devices in order to further investigate on the first part of the research question. This will be helpful in generating a wider data set which intern increase the accuracy of the final result.

Further research can be carried out to implement this hybrid network completely in the kernel layer of the Android device to improve the performance further. 

As this is a novel solution the routing algorithm for this hybrid network can also be improved in order to give more efficient routing solutions for the end devices by the control plane. 

The current implementation described in chapter 5 uses a centralized system for the control plane which leads to problems in the availability of the system. Finding the feasibility of implementing a highly available decentralized control plane can prove further in time. 
